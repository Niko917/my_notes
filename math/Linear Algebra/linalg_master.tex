\documentclass[12px]{report}
\usepackage[russian]{babel}
\usepackage{hyphenat}
\usepackage{xparse}
\usepackage{hyperref}
\usepackage{geometry}
\usepackage{import}
\usepackage{xifthen}
\usepackage{pdfpages}
\usepackage{transparent}
\usepackage{graphicx}
\graphicspath{ {./figures/} }
\usepackage{xargs}
\usepackage{parskip}
\usepackage{marginnote}
\usepackage{todonotes}
\usepackage{amsmath, amssymb, amsthm}
\usepackage{mathtools}
\usepackage{centernot}
\usepackage{bm}
\usepackage{thmtools}
\usepackage{mdframed}
\usepackage{subcaption}
\usepackage{titlesec}
\usepackage{enumitem}
\usepackage{xecyr}
\usepackage[russian]{cleveref}  % after amsmath
\usepackage{accents}
\usepackage{fancyhdr}
\pagestyle{fancy}
\usepackage{xcolor}
\usepackage{amsthm}
\usepackage{shadethm}
\usepackage{framed}

\newcommand*\mnote[1]{\marginnote{#1}}

% SETTINGS
\geometry{
  a4paper,
  left=20mm,
  right=35mm,  % for margin notes
  top=25mm,
  bottom=25mm
}

\newcommand\N{\ensuremath{\mathbb{N}}}
\newcommand\R{\ensuremath{\mathbb{R}}}
\newcommand\Z{\ensuremath{\mathbb{Z}}}
\renewcommand\O{\ensuremath{\varnothing}}
\newcommand\Q{\ensuremath{\mathbb{Q}}}

\DeclareMathOperator{\End}{End}
\DeclareMathOperator{\dom}{dom}
\DeclareMathOperator{\rnd}{rnd}
\DeclareMathOperator{\id}{id}
\DeclareMathOperator{\Imf}{Im}
\DeclareMathOperator{\Ker}{Ker}
\DeclareMathOperator{\Frac}{Frac}
\DeclareMathOperator{\ord}{ord}
\DeclareMathOperator{\ev}{ev}
\DeclareMathOperator{\Char}{Char}
\DeclareMathOperator{\Hom}{Hom}

\def\@lecture{}%
\newcommand{\lecture}[3]{
    \ifthenelse{\isempty{#3}}{%
        \def\@lecture{Лекция #1}%
    }{%
        \def\@lecture{Лекция #1: #3}%
    }%
    \subsection*{\@lecture}
    \marginpar{\small\textsf{\mbox{#2}}}
}

\usepackage{amsthm}
\usepackage{xcolor}
\usepackage{framed}

% Определение цветов
\definecolor{thmcolor}{HTML}{EDF8FF} % Светло-голубой
\definecolor{defcolor}{HTML}{F3E5F5} % Светло-голубой
\definecolor{excolor}{HTML}{FFF8DC} % Светло-коричневый
\definecolor{propcolor}{HTML}{F0FFF0} % Светло-зеленый
\definecolor{stmtcolor}{HTML}{FFF0F5} % Светло-розовый
\definecolor{lemcolor}{HTML}{F5FFFA} % Светло-зеленый

% Определение стилей
\newtheoremstyle{mythmstyle}
  {3pt} % Space above
  {3pt} % Space below
  {} % Body font
  {} % Indent amount
  {\bfseries} % Theorem head font
  {.} % Punctuation after theorem head
  {.5em} % Space after theorem head
  {} % Theorem head spec (can be left empty, meaning `normal')

\newtheoremstyle{mydefstyle}
  {3pt} % Space above
  {3pt} % Space below
  {} % Body font
  {} % Indent amount
  {\bfseries} % Theorem head font
  {.} % Punctuation after theorem head
  {.5em} % Space after theorem head
  {} % Theorem head spec (can be left empty, meaning `normal')

\newtheoremstyle{myexstyle}
  {3pt} % Space above
  {3pt} % Space below
  {} % Body font
  {} % Indent amount
  {\bfseries} % Theorem head font
  {.} % Punctuation after theorem head
  {.5em} % Space after theorem head
  {} % Theorem head spec (can be left empty, meaning `normal')

\newtheoremstyle{mypropstyle}
  {3pt} % Space above
  {3pt} % Space below
  {} % Body font
  {} % Indent amount
  {\bfseries} % Theorem head font
  {.} % Punctuation after theorem head
  {.5em} % Space after theorem head
  {} % Theorem head spec (can be left empty, meaning `normal')

\newtheoremstyle{mystmtstyle}
  {3pt} % Space above
  {3pt} % Space below
  {} % Body font
  {} % Indent amount
  {\bfseries} % Theorem head font
  {.} % Punctuation after theorem head
  {.5em} % Space after theorem head
  {} % Theorem head spec (can be left empty, meaning `normal')

\newtheoremstyle{mylemstyle}
  {3pt} % Space above
  {3pt} % Space below
  {} % Body font
  {} % Indent amount
  {\bfseries} % Theorem head font
  {.} % Punctuation after theorem head
  {.5em} % Space after theorem head
  {} % Theorem head spec (can be left empty, meaning `normal')

% Применение стилей
\theoremstyle{mythmstyle}
\newtheorem{theorem}{Теорема}

\theoremstyle{mydefstyle}
\newtheorem{definition}{Определение}

\theoremstyle{myexstyle}
\newtheorem{example}{Пример}

\theoremstyle{mypropstyle}
\newtheorem{property}{Свойство}

\theoremstyle{mystmtstyle}
\newtheorem{statement}{Утверждение}

\theoremstyle{mylemstyle}
\newtheorem{lemma}{Лемма}

% Оформление фреймов
\newenvironment{shth}
  {\def\FrameCommand{\fboxsep=\FrameSep \fcolorbox{black}{thmcolor}}\MakeFramed {\FrameRestore}}
  {\endMakeFramed}

\newenvironment{shdef}
  {\def\FrameCommand{\fboxsep=\FrameSep \fcolorbox{black}{defcolor}}\MakeFramed {\FrameRestore}}
  {\endMakeFramed}

\newenvironment{shex}
  {\def\FrameCommand{\fboxsep=\FrameSep \fcolorbox{black}{excolor}}\MakeFramed {\FrameRestore}}
  {\endMakeFramed}

\newenvironment{shprop}
  {\def\FrameCommand{\fboxsep=\FrameSep \fcolorbox{black}{propcolor}}\MakeFramed {\FrameRestore}}
  {\endMakeFramed}

\newenvironment{shstmt}
  {\def\FrameCommand{\fboxsep=\FrameSep \fcolorbox{black}{stmtcolor}}\MakeFramed {\FrameRestore}}
  {\endMakeFramed}

\newenvironment{shlem}
  {\def\FrameCommand{\fboxsep=\FrameSep \fcolorbox{black}{lemcolor}}\MakeFramed {\FrameRestore}}
  {\endMakeFramed}

\theoremstyle{plain}
\renewcommand\qedsymbol{$\blacksquare$}

\declaretheoremstyle[
  numbered=no,
  qed=$\lrcorner$
]{mystyle}

% Настройка заголовков
\titleformat{\section}
  {\normalfont\Large\bfseries}{\thesection}{1em}{}
\titleformat{\subsection}
  {\normalfont\large\bfseries}{\thesubsection}{1em}{}
\titleformat{\subsubsection}
  {\normalfont\normalsize\bfseries}{\thesubsubsection}{1em}{}

% Убираем отображение названия главы вверху страницы
\titlespacing*{\section}{0pt}{3.5ex plus 1ex minus .2ex}{2.3ex plus .2ex}
\titlespacing*{\subsection}{0pt}{3.25ex plus 1ex minus .2ex}{1.5ex plus .2ex}
\titlespacing*{\subsubsection}{0pt}{3.25ex plus 1ex minus .2ex}{1.5ex plus .2ex}

% Настройка верхнего колонтитула
\fancyhf{} % Очищаем верхний и нижний колонтитулы
\renewcommand{\headrulewidth}{0pt} % Убираем линию вверху страницы
\fancyfoot[C]{\thepage} % Номер страницы внизу по центру



\title{Дополнительные главы линейной алгебры}
\author{составлено Николаем Колбом}

\begin{document}
  \maketitle
  \tableofcontents

  \chapter{Евклидовы и унитарные пространства}

  \section{Билинейные формы и их свойства}

\large Пусть $V$ - Линейное пространство над вещественным полем $\mathbb{R}$

\vspace{0.5cm}


\begin{shdef}
    \begin{definition}
        \leavevmode\newline
        
        \textbf{Билинейной формой} на $V$ называется функция $f(x,y): V \times V \to \mathbb{R}$, для которой определены следующие аксиомы:
        \begin{enumerate}
            \item $\forall x,y,z \in V \quad \; f(x+y,z) = f(x, z) + f(y,z)$
            \item $\forall \alpha \in \R, \forall x,y \in V \quad \;  f(\alpha x, y) = \alpha f(x,y)$
            \item $\forall x, y, z \in V \quad \; f(x,y,z) = f(x,y) + f(x,z)$
            \item $\forall \beta \in \R, \forall x, y \in V \quad \; f(x, \beta y) = \beta f(x,y)$
        \end{enumerate}
    \end{definition}
\end{shdef}

\vspace{0.5cm}
\textbf{Свойства билинейной формы:}

\begin{itemize}
    \item Матричный вид билинейной формы $f$ в базисе $E$ задается матрицей $A = (a_{ij})$, где $a_{ij} = f(e_i, e_j)$. 

    Тогда
    \[
    f(x, y) = x^T A y
    \]
    где $x$ и $y$ — вектор-столбцы координат $x$ и $y$ в базисе $E$.
    \item Билинейная форма $f$ называется \underline{симметричной}, если
    \[
    f(u, v) = f(v, u) \quad \forall u, v \in V
    \]
    \item Билинейная форма $f$ называется \underline{кососимметричной}, если
    \[
    f(u, v) = -f(v, u) \quad \forall u, v \in V
    \]
    \item Билинейная форма $f$ называется \underline{невырожденной}, если для любого \newline ненулевого вектора $u \in V$ существует вектор $v \in V$ такой, что $f(u, v) \neq 0$.
\end{itemize}

\clearpage

Пусть $V$ — векторное пространство над полем $\mathbb{C}$.
\vspace{0.5cm}
\begin{shdef}
    \begin{definition}
    \leavevmode \newline
    
    \textbf{Полуторолинейной формой} на $V$ называется функция 
$g: V \times V \to \mathbb{C}$, \newline удовлетворяющая следующим аксиомам:
\begin{enumerate}
    \item $\forall x,y,z \in V \quad g(x + y,z) = g(x,z) + g(y,z)$
    \item $\forall \alpha \in \mathbb{C} \; \forall x,y \in V \quad  g(\alpha x, y) = \alpha g(x,y)$
    \item $\forall x,y,z \in V \quad g(x,y+z) = g(x,y) + g(x,z)$
    \item $\forall \mu \in \mathbb{C}, \; \forall x,y \in V \quad  g(x, \mu y) = \overline{\mu} g(x,y)$
    \item $g(x,y) = \sum_{i,j = 1}^{n} \ g(e_{i}, e_{j}) \ \xi_{i} \ \overline{\nu_{j}}$
\end{enumerate}
    \end{definition}
\end{shdef}

\vspace{0.5cm}

\section{Определения евклидовых и унитарных пространств}

\begin{shdef}
    \begin{definition}
        \leavevmode\newline
        
        \textbf{Евклидово пространство} — это вещественное линейное пространство $E$, на котором задана симметричная положительно определенная билинейная форма $\langle \cdot, \cdot \rangle: E \times E \to \mathbb{R}$, обладающая следующими свойствами:
        \begin{enumerate}
            \item $\forall x, y \in E \quad \langle x, y \rangle = \langle y, x \rangle$
            \item $\forall \alpha \in E \quad \forall x, y \in E \quad \langle \lambda x, y \rangle = \alpha \langle x, y \rangle$
            \item $\forall x, y, z \in E \quad \langle x + y,z \rangle = \langle x,z \rangle + \langle y,z \rangle)$
            \item $\forall x \in E \quad \langle x, x \rangle \geq 0 \quad (\langle x,x \rangle = 0 \Longleftrightarrow x = 0)$
        \end{enumerate}
        \vspace{0.3cm}
        \textbf{Из свойств (1, 2, 3) вытекает следствие:} $\langle x, \alpha y + \beta z \rangle = \alpha \langle x,y \rangle + \beta \langle x,z \rangle$
    \end{definition}
\end{shdef}
\vspace{2.0cm}
Таким образом, некоторое пространство $E$ считается евклидовым, если и только если на нём определена скалярная симметричная билинейная форма, называемая \textbf{скалярным произведением}.

\clearpage
\textbf{Примеры евклидовых пространств:}
\begin{shex}
    \begin{enumerate}
        \item
        Пусть $E = \mathbb{R}^n$.
        Скалярное произведение в $\mathbb{R}^n$ можно записать в виде:
        \[
        \langle x, y \rangle = \sum_{i=1}^n \xi_i \mu_i
        \]
        где $x = (\xi_1, \xi_2, \ldots, \xi_n)$ и $y = (\mu_1, \mu_2, \ldots, \mu_n)$ — векторы в $\mathbb{R}^n$
        
        \clearpage
        \item
        Пусть $E$ — пространство непрерывных функций на отрезке $[a, b]$.
        \newline 
        Скалярное произведение можно определить как:
        \[
        \langle f, g \rangle = \int_a^b f(x) g(x) \, dx
        \]
        
    \end{enumerate}
\end{shex}

\vspace{1cm}

\begin{shdef}
    \begin{definition}
        \leavevmode \newline
        
        \textbf{Унитарное пространство} — это комплексное линейное пространство $U$, на котором задана эрмитова положительно определенная полуторалинейная форма $\langle \cdot, \cdot \rangle: U \times U \to \mathbb{C}$, обладающая следующими свойствами:
    \begin{enumerate}
        \item $\forall x, y \in U \quad \langle x, y \rangle = \overline{\langle y, x \rangle}$
        \item $\forall \lambda \in \mathbb{C} \quad \forall x, y \in U \quad \langle \lambda x, y \rangle = \lambda \langle x, y \rangle$
        \item $\forall x, y, z \in U \quad \langle x + y, z \rangle = \langle x, z \rangle + \langle y, z \rangle$
        \item $\forall x \in U \quad \langle x,x \rangle \in \R \quad \langle x, x \rangle \geq 0 \quad (\langle x, x \rangle = 0 \Longleftrightarrow x = 0)$
    \end{enumerate}
    \vspace{0.3cm}
    \end{definition}
\end{shdef}


\vspace{1.0cm}
\textbf{Примеры унитарных пространств:}
\begin{shex}
    \begin{enumerate}
        \item Пусть \( U = \mathbb{C}^n \).
    Скалярное произведение в \(\mathbb{C}^n\) можно записать в виде:
    \[
    \langle x, y \rangle = \sum_{i=1}^n \xi_i \overline{\mu_i}
    \]
    где \( x = (\xi_1, \xi_2, \ldots, \xi_n) \) и \( y = (\mu_1, \mu_2, \ldots, \mu_n) \) — векторы в \(\mathbb{C}^n\).

\vspace{0.2cm}
        \item Пусть \( U \) — пространство непрерывных функций на отрезке \([a, b]\) с комплексными значениями.
    Скалярное произведение можно определить как:
    \[
    \langle f, g \rangle = \int_a^b f(x) \overline{g(x)} \, dx
    \]
    \end{enumerate}
\end{shex}


\clearpage
Пусть $V -$ Евклидово / Унитарное пространство.

\begin{shdef}
    \begin{definition}
    \leavevmode \newline
   
    $\|x\| = \sqrt{\langle x, x \rangle}$ - норма вектора $x \in V$
        \begin{itemize}
            \item $\|x\| > 0 \quad \longleftrightarrow  \quad x \neq 0$
            \item  $\|x\| = 0 \quad \longleftrightarrow \quad x = 0$
        \end{itemize}
    \end{definition}
\end{shdef}

\vspace{1.0cm}
\textbf{Примеры норм:}
\begin{shex}
    \begin{itemize}
        \item Класс Гёльдоровых норм ($\|X\|_p = (|x_{1}|^p + |x_{2}|^p + \ldots |x_{n}^p|)^\frac{1}{p}$), \quad где $p \geq 1$.
        \begin{enumerate}
            \item \textbf{Евклидова норма в $\R^n$:} $\quad \|x\| = \sqrt{\xi_{1}^2 + \ldots \xi_{n}^2} = \sqrt{\sum_{k = 1}^{n} \xi_{k}^2}$
            \item \textbf{Евклидова норма в $\mathbb{C}^n$:} $\quad \|x\| = \sqrt{|\xi_{1}|^2 + \ldots + |\xi_{n}|^2}$
        \end{enumerate}
    \end{itemize}
\end{shex}



\clearpage
\section{Неравенство Коши-Буняковского}
\begin{shth}
\begin{theorem}
    (Неравенство Коши-Буняковского).
    \newline
    
    \vspace{0.3cm}
    Пусть $K -$ это $E$ или $V$, \quad $x,y \in K$, \quad тогда $\| \langle x,y \rangle\| \; \leq \; \|x\| \cdot \|y\|$
\end{theorem}
\end{shth}

\vspace{0.2cm}
\begin{proof}



    \begin{enumerate}
        \item 
        
        $\textbf{Для Евклидова пространства} \; E:$ \; $\forall x, y \in E, \quad \forall \alpha \in \mathbb{R}$
        
        $$0 \; \leq \; \langle x \; - \; \alpha y, x \; - \; \alpha y \rangle \; = \; \langle x, x \rangle \; - \; \alpha \langle y, x \rangle \; - \; \alpha \langle x, y \rangle \; + \; \alpha^2 \langle y, y \rangle \; = \; \|x\|^2 \; - \; 2 \alpha \langle x, y \rangle \; + \; \alpha^2 \|y\|^2 \; \geq \; 0$$
        
        Это квадратный трёхчлен относительно \( \alpha \). 
        
        Для того чтобы оно выполнялось для всех \( \alpha \), дискриминант должен быть не положительным:
        
        $$\frac{D}{4} = \langle x, y \rangle^2 - \|x\|^2 \|y\|^2 \leq 0$$
        
        
        $$\Longrightarrow \quad |\langle x, y \rangle| \leq \|x\| \|y\|$$
        
    \item \textbf{Для унитарного пространства \( V \):} \quad $\forall x, y \in V, \quad \forall \alpha, \beta \in \mathbb{C}$.
        
        $$0 \; \leq \; \langle \alpha x \; + \; \beta y, \alpha x \; + \; \beta y \rangle \; = \; \alpha \overline{\alpha} \langle x, x \rangle \; + \; \alpha \overline{\beta} \langle x, y \rangle \; + \; \beta \overline{\alpha} \langle y, x \rangle \; + \; \beta \overline{\beta} \langle y, y \rangle \; =$$
        
        $$= |\alpha|^2 \; \|x\|^2 \; + \; \alpha \overline{\beta} \langle x, y \rangle \; + \; \beta \; \overline{\alpha} \overline{\langle x, y \rangle } \; + \; |\beta|^2 \; \|y\|^2 \qquad \Longrightarrow$$
        
        $$\alpha := \|y\|^2 ) , \quad ( \beta := -\langle x, y \rangle$$
 
        $$\Longrightarrow \quad \|y\|^4 \; \|x\|^2 - \|y\|^2 \langle x, y \rangle \overline{\langle x, y \rangle} - \|y\|^2 \; \langle x, y \rangle \overline{\langle x, y \rangle} \; + \; |\langle x, y \rangle |^2 \; \|y\|^2 \; =$$
        
        $$= \; \|y\|^2 \; (\|y\|^2 \; \|x\|^2 \; - \; \langle x, y \rangle \overline{\langle x, y \rangle} \; - \; \langle x,y \rangle \overline{\langle x, y \rangle} \; + \; |\langle x, y \rangle|^2) \; = \; \|y\|^2(\|x\|^2 \; \|y\|^2 - |\langle x, y \rangle|^2) \; \geq \; 0$$
        
        
        \begin{itemize}
        
        \vspace{0.4cm} \item $y \; \neq \; 0 \; \Longrightarrow \; \|x\|^2 \; \|y\|^2 - |\langle x, y \rangle|^2 \; \geq \; 0 \; \Longrightarrow |\langle x, y \rangle| \; \leq \; \|x\| \; \|y\|$
        
        \vspace{0.4cm} \item $y \; = \; 0 \; \Longrightarrow \|y\| \; = \; 0, \; \langle x, y \rangle \; = \; 0 \; \Longrightarrow \; |\langle x, y \rangle \; \leq \; \|x\| \; \|y\|$
        \end{itemize}
    \end{enumerate}
\end{proof}

\clearpage
\section{Неравенство Минковского}
\begin{shth}
    \begin{theorem}
        (Неравенство Минковского $\triangle$).
        \newline
        $\forall x,y \in V \quad ||x + y|| \; \leq \; ||x|| + ||y||$
    \end{theorem}
\end{shth}


\vspace{0.2cm}
\begin{proof}
    \begin{enumerate}
        \item \textbf{Для Евклидова пространства \( E \):} \quad \(\forall x, y \in E\)
        
        \[
        \|x + y\|^2 = \langle x + y, x + y \rangle = \langle x, x \rangle + \langle x, y \rangle + \langle y, x \rangle + \langle y, y \rangle
        \]
        
        Используя неравенство Коши-Буняковского, получаем:
        
        \[
        \langle x, y \rangle + \langle y, x \rangle \leq 2 \cdot \frac{|\langle x, y \rangle| + |\langle y, x \rangle|}{2} \leq 2 \cdot \frac{\|x\| \|y\| + \|y\| \|x\|}{2} = 2 \|x\| \|y\|
        \]
        
        Таким образом:
        
        \[
        \|x + y\|^2 \leq \|x\|^2 + 2 \|x\| \|y\| + \|y\|^2 = (\|x\| + \|y\|)^2
        \]
        
        Извлекая квадратный корень из обеих частей, получаем:
        
        \[
        \|x + y\| \leq \|x\| + \|y\|
        \]
        
        \item \textbf{Для унитарного пространства \( V \):} \quad \(\forall x, y \in V\)
        
        \[
        \|x + y\|^2 = \langle x + y, x + y \rangle = \langle x, x \rangle + \langle x, y \rangle + \langle y, x \rangle + \langle y, y \rangle
        \]
        
        Используя неравенство Коши-Буняковского, получаем:
        
        \[
        ||x||^2 \; + \; \overline{\langle x, y \rangle} \; + \; \langle x ,y \rangle \; + \; ||y||^2 \; \leq \; ||x||^2 \; + \; 2 |\langle x,y \rangle | \; + \; ||y||^2  \; \leq \newline 
        \]
        
        $$\; \leq \; ||x||^2 \; + \; 2 ||x|| ||y|| \; + \; ||y||^2 = \langle ||x|| \; + \; ||y|| \rangle^2$$
        
        
        Извлекая квадратный корень из обеих частей, получаем:
        
        
        $$\|x + y\| \leq \|x\| + \|y\|$$
        
   
        
    \end{enumerate}
\end{proof}

\begin{shex}
    Примеры:
    \begin{enumerate}
        \item $\R^n: \quad \langle x, y \rangle = \xi_1 \mu_1 + \ldots + \xi_n \mu_n \quad (\sum\limits_{i = 1}^{n} \xi_i \; \mu_i)^2 \; \leq \; (\sum\limits_{i = 1}^{n} \xi_{i}^2) \; (\sum\limits_{i = 1}^{n} \mu_{i}^2)$
        \item $\mathbb{C}^n: \quad \langle x,y \rangle = \xi_{1} \overline{\mu_{1}} + \ldots + \xi_{n} \overline{\mu_{n}} \quad (\sum\limits_{i = 1}^{n} \xi_{i} \; \overline{\mu_{i}}) \; \leq \; (\sum\limits_{i = 1}^{n} |\xi_{i}|^2) \; (\sum\limits_{i = 1}^{n} |\mu_{i}|^2)$
    \end{enumerate}
\end{shex}

\clearpage
\begin{shdef}
\begin{definition}
    \leavevmode \newline
    Пусть $\mathbb{V}$ — это $E$ или $V$, и $\mathbb{E} = \{e_1, \ldots, e_n\}$ — базис в $\mathbb{V}$. \newline
    Матрицей Грама называется матрица $G$, элементы которой определяются как:
    \[
    G_{ij} = (\langle e_i, e_j \rangle),
    \]
    где $\langle \cdot, \cdot \rangle$ обозначает скалярное произведение в $\mathbb{V}$.
\end{definition}
\end{shdef}

\vspace{0.2cm}

\begin{shth}
    \begin{theorem}
        Теперь $\mathbb{V}$ - это только $E$. \newline
        $a_{1}, \ldots, a_{k} \in E, \quad G = (\langle a_{i} a_{j} \rangle)$
        
        \begin{enumerate}
            \item Если ${a_{1}, \ldots, a_{k}}$ - ЛНЗ, то $|G| > 0$
            \item Если ${a_{1}, \ldots, a_{k}}$ - ЛЗ, то $|G| = 0$
        \end{enumerate}
    \end{theorem}
\end{shth}

\vspace{0.2cm}
\begin{proof}
\leavevmode \newline
    \begin{enumerate}
        \item ${a_{1}, \ldots, a_{k}}$ - ЛНЗ, \quad $E_{1} = \alpha(a_{1}, \ldots, a_{k})$ 
        \newline 
        
        $\forall x, y \in E_{1}$ \quad  $x = \xi_{1} a_{1} \; + \; \ldots \; + \; \xi_{k} a_{k}, \quad y = \mu_{1} a_{1} \; + \; \ldots \; + \; \mu_{k} a_{k}$ 
        \newline 
        
        $\langle x, y \rangle = \sum\limits_{i, j = 1}^{k} (a_{i} a_{j}) \xi_{i} \mu_{j} \Longrightarrow$ по критерию Сильвестра $|E| > 0$
        \newline 
        
        \item ${a_{1}, \ldots, a_{k}}$ - ЛЗ \; $\Longrightarrow \exists \alpha_{1}, \ldots, \alpha_{k}$ не все нули, такие что $\alpha_{1} a_{1} + \ldots + \alpha_{k} a_{k} = \boldsymbol{0}$
        \newline
        
        
    \[
    \Longrightarrow
        \begin{cases}
            \alpha_{1} \langle a_1, a_{1} \rangle + \ldots + \alpha_{k} \langle a_{k}, a_{1} \rangle = \mathbf{0} \\
            \alpha_{1} \langle a_{1}, a_{2} \rangle + \ldots + \alpha_{k} \langle a_{k}, a_{2} \rangle = \mathbf{0} \\
            \vdots \\
            
            \alpha_{1} \langle a_{1}, a_{k} \rangle + \ldots + \alpha_{k} \langle a_{k}, a_{k} \rangle = \mathbf{0}
        \end{cases}
    \]
    \newline
    
    $\Longrightarrow$ Система имеет нетривиальное решение $\Longrightarrow$ её определитель равен нулю, а это и есть $|G|.$
    \end{enumerate}
\end{proof}

\clearpage

\section{Ортонормированные базисы в $E$ и $V$}

$\mathbb{V}$ - это $E$ или $V$

\begin{shdef}
    \begin{definition}
        \
        $x \perp y$, если $\langle x, y \rangle = 0$
    \end{definition}
    
    \begin{definition}
        $\mathbb{E}$ - базис в $\mathbb{V}$. \quad
        \
        $\mathbb{E}$ - ОНБ, \ если $\langle e_{i}, e_{j} \rangle = \delta_{ij} = \begin{cases}
        1, i = j \\
        0, i \neq j \\
        \end{cases}$
    \end{definition}
\end{shdef}

\vspace{0.4cm}
\begin{shth}
    \begin{theorem}
        \leavevmode \newline
        Если $a_{1}, \ldots, a_{k} \neq 0$ и $a_{i} \perp a_{j}, \ i \neq j,$ то это ЛНЗ система.
    \end{theorem}
\end{shth}

\begin{proof}
\leavevmode \newline

    От противного: Они ЛЗ $\Longrightarrow$ $\exists \ \alpha_{1}, \ldots, \alpha_{k}$ не все 0, что: 
\newline 

    $$(*) \ \alpha_{1} a_{1} + \ldots + \alpha_{k} a_{k} = 0$$.
    Умножаем скалярно на $a_{1}$
    \newline
    
    $\Longrightarrow \alpha_{1} \langle a_{1}, a_{1} \rangle + \alpha_{2} \langle a_{2}, a_{1} \rangle + \ldots + \alpha_{k} \langle a_{k}, a_{1} \rangle = 0 \ \Longrightarrow \alpha_{1} = 0$
    \newline
    
    умножаем $(*)$ на $a_{2}$ и получаем, что $\alpha_{2} = 0$
    
    \
    
    $\Longrightarrow$ все $\alpha_{i} = 0 \ \Longrightarrow$ противоречие! 
\end{proof}

\vspace{0.4cm}

\begin{shth}
    \begin{theorem}
        (об ортогонализации по Шмидту)

        \end{enumerate}
    \end{theorem}
\end{shth}
\end{document}
