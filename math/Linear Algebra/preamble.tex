\usepackage[russian]{babel}
\usepackage{hyphenat}
\usepackage{xparse}
\usepackage{hyperref}
\usepackage{geometry}
\usepackage{import}
\usepackage{xifthen}
\usepackage{pdfpages}
\usepackage{transparent}
\usepackage{graphicx}
\graphicspath{ {./figures/} }
\usepackage{xargs}
\usepackage{parskip}
\usepackage{marginnote}
\usepackage{todonotes}
\usepackage{amsmath, amssymb, amsthm}
\usepackage{mathtools}
\usepackage{centernot}
\usepackage{bm}
\usepackage{thmtools}
\usepackage{mdframed}
\usepackage{subcaption}
\usepackage{titlesec}
\usepackage{enumitem}
\usepackage{xecyr}
\usepackage[russian]{cleveref}  % after amsmath
\usepackage{accents}
\usepackage{fancyhdr}
\pagestyle{fancy}
\usepackage{xcolor}
\usepackage{amsthm}
\usepackage{shadethm}
\usepackage{framed}
\usepackage{tikz}
\usetikzlibrary{calc}
\usetikzlibrary{arrows.meta, positioning}


\newcommand*\mnote[1]{\marginnote{#1}}

% SETTINGS
\geometry{
  a4paper,
  left=20mm,
  right=35mm,  % for margin notes
  top=25mm,
  bottom=25mm
}

\newcommand\N{\ensuremath{\mathbb{N}}}
\newcommand\R{\ensuremath{\mathbb{R}}}
\newcommand\Z{\ensuremath{\mathbb{Z}}}
\renewcommand\O{\ensuremath{\varnothing}}
\newcommand\Q{\ensuremath{\mathbb{Q}}}
\newcommand{\E}{\mathcal{E}}
\newcommand{\E'}{\mathcal{E'}}

\DeclareMathOperator{\End}{End}
\DeclareMathOperator{\dom}{dom}
\DeclareMathOperator{\rnd}{rnd}
\DeclareMathOperator{\id}{id}
\DeclareMathOperator{\Imf}{Im}
\DeclareMathOperator{\Ker}{Ker}
\DeclareMathOperator{\Frac}{Frac}
\DeclareMathOperator{\ord}{ord}
\DeclareMathOperator{\ev}{ev}
\DeclareMathOperator{\Char}{Char}
\DeclareMathOperator{\Hom}{Hom}

\def\@lecture{}%
\newcommand{\lecture}[3]{
    \ifthenelse{\isempty{#3}}{%
        \def\@lecture{Лекция #1}%
    }{%
        \def\@lecture{Лекция #1: #3}%
    }%
    \subsection*{\@lecture}
    \marginpar{\small\textsf{\mbox{#2}}}
}

\usepackage{amsthm}
\usepackage{xcolor}
\usepackage{framed}

% Определение цветов
\definecolor{thmcolor}{HTML}{EDF8FF} % Светло-голубой
\definecolor{defcolor}{HTML}{F3E5F5} % Светло-голубой
\definecolor{excolor}{HTML}{FFF8DC} % Светло-коричневый
\definecolor{propcolor}{HTML}{F0FFF0} % Светло-зеленый
\definecolor{stmtcolor}{HTML}{FFF0F5} % Светло-розовый
\definecolor{lemcolor}{HTML}{F5FFFA} % Светло-зеленый

% Определение стилей
\newtheoremstyle{mythmstyle}
  {3pt} % Space above
  {3pt} % Space below
  {} % Body font
  {} % Indent amount
  {\bfseries} % Theorem head font
  {.} % Punctuation after theorem head
  {.5em} % Space after theorem head
  {} % Theorem head spec (can be left empty, meaning `normal')

\newtheoremstyle{mydefstyle}
  {3pt} % Space above
  {3pt} % Space below
  {} % Body font
  {} % Indent amount
  {\bfseries} % Theorem head font
  {.} % Punctuation after theorem head
  {.5em} % Space after theorem head
  {} % Theorem head spec (can be left empty, meaning `normal')

\newtheoremstyle{myexstyle}
  {3pt} % Space above
  {3pt} % Space below
  {} % Body font
  {} % Indent amount
  {\bfseries} % Theorem head font
  {.} % Punctuation after theorem head
  {.5em} % Space after theorem head
  {} % Theorem head spec (can be left empty, meaning `normal')

\newtheoremstyle{mypropstyle}
  {3pt} % Space above
  {3pt} % Space below
  {} % Body font
  {} % Indent amount
  {\bfseries} % Theorem head font
  {.} % Punctuation after theorem head
  {.5em} % Space after theorem head
  {} % Theorem head spec (can be left empty, meaning `normal')

\newtheoremstyle{mystmtstyle}
  {3pt} % Space above
  {3pt} % Space below
  {} % Body font
  {} % Indent amount
  {\bfseries} % Theorem head font
  {.} % Punctuation after theorem head
  {.5em} % Space after theorem head
  {} % Theorem head spec (can be left empty, meaning `normal')

\newtheoremstyle{mylemstyle}
  {3pt} % Space above
  {3pt} % Space below
  {} % Body font
  {} % Indent amount
  {\bfseries} % Theorem head font
  {.} % Punctuation after theorem head
  {.5em} % Space after theorem head
  {} % Theorem head spec (can be left empty, meaning `normal')

% Применение стилей
\theoremstyle{mythmstyle}
\newtheorem{theorem}{Теорема}

\theoremstyle{mydefstyle}
\newtheorem{definition}{Определение}

\theoremstyle{myexstyle}
\newtheorem{example}{Пример}

\theoremstyle{mypropstyle}
\newtheorem{property}{Свойство}

\theoremstyle{mystmtstyle}
\newtheorem{statement}{Утверждение}

\theoremstyle{mylemstyle}
\newtheorem{lemma}{Лемма}


\newcommand{\norm}[1]{\|#1\|}
\newcommand{\inner}[2]{\langle #1, #2 \rangle}
\newcommand{\nl}{\newline}
\newcommand{\lra}{\Longrightarrow}
\newcommand{\lla}{\Longleftarrow}
\newcommand{\tt}{\text}

% Оформление фреймов
\newenvironment{shth}
  {\def\FrameCommand{\fboxsep=\FrameSep \fcolorbox{black}{thmcolor}}\MakeFramed {\FrameRestore}}
  {\endMakeFramed}

\newenvironment{shdef}
  {\def\FrameCommand{\fboxsep=\FrameSep \fcolorbox{black}{defcolor}}\MakeFramed {\FrameRestore}}
  {\endMakeFramed}

\newenvironment{shex}
  {\def\FrameCommand{\fboxsep=\FrameSep \fcolorbox{black}{excolor}}\MakeFramed {\FrameRestore}}
  {\endMakeFramed}

\newenvironment{shprop}
  {\def\FrameCommand{\fboxsep=\FrameSep \fcolorbox{black}{propcolor}}\MakeFramed {\FrameRestore}}
  {\endMakeFramed}

\newenvironment{shstmt}
  {\def\FrameCommand{\fboxsep=\FrameSep \fcolorbox{black}{stmtcolor}}\MakeFramed {\FrameRestore}}
  {\endMakeFramed}

\newenvironment{shlem}
  {\def\FrameCommand{\fboxsep=\FrameSep \fcolorbox{black}{lemcolor}}\MakeFramed {\FrameRestore}}
  {\endMakeFramed}

\theoremstyle{plain}
\renewcommand\qedsymbol{$\blacksquare$}

\declaretheoremstyle[
  numbered=no,
  qed=$\lrcorner$
]{mystyle}

% Настройка заголовков
\titleformat{\section}
  {\normalfont\Large\bfseries}{\thesection}{1em}{}
\titleformat{\subsection}
  {\normalfont\large\bfseries}{\thesubsection}{1em}{}
\titleformat{\subsubsection}
  {\normalfont\normalsize\bfseries}{\thesubsubsection}{1em}{}

% Убираем отображение названия главы вверху страницы
\titlespacing*{\section}{0pt}{3.5ex plus 1ex minus .2ex}{2.3ex plus .2ex}
\titlespacing*{\subsection}{0pt}{3.25ex plus 1ex minus .2ex}{1.5ex plus .2ex}
\titlespacing*{\subsubsection}{0pt}{3.25ex plus 1ex minus .2ex}{1.5ex plus .2ex}

% Настройка верхнего колонтитула
\fancyhf{} % Очищаем верхний и нижний колонтитулы
\renewcommand{\headrulewidth}{0pt} % Убираем линию вверху страницы
\fancyfoot[C]{\thepage} % Номер страницы внизу по центру

